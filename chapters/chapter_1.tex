%\section{Introduction}
\section{Data Acquisition System} 
In recent years, a number of sectors, including process, fabrication, and automation, have adopted the practice of measuring, controlling, and testing equipment and processes utilizing computer systems and microprocessors. One such field that has been reinvented by the PC's rapid technical growth is data acquisition (DAQ), commonly known as DAQ. In order to collect, measure, store, and analyze data from diverse sources, a data acquisition system (DAS) combines hardware and software components. It makes it possible to gather actual signals or data from the outside environment and transforms it into a digital format for processing, analysis, and display. DAQ gear often handles the conversion of analog to digital signals as well as the reverse. The DAQ technology often works alongside a computer, such a standard PC. In this example, the computer is running DAQ software, which records and examines the data.



\vspace{0.5cm}
Data collection systems have evolved throughout time from electromechanical records with one to four channels to completely electronic architectures that can simultaneously measure hundreds of variables. In the early structures, the signals were continually recorded on paper graphs or magnetic tape rolls, but following the introduction of computers, especially personal ones, both the volume and speed of data collection substantially improved. The majority of conventional datalogger systems are still in use today, nevertheless. From straightforward setups with a few channels to complicated systems with hundreds or even thousands of channels, high-speed sampling rates, and networked architectures, data collecting systems can range in complexity and scalability.


\section{Core components of DAQ }
\textbf{Sensors and Transducers:}
Such devices soak up electrical or physical occurrences and turn them into signals. They might be flow meters, strain gauges, accelerometers, temperature sensors, pressure transducers, and many more.
\vspace{0.5cm}\par
\textbf{Signal Conditioning:}
It is frequently necessary to condition the sensor-acquired signals in order to assure their correctness, stability, and compatibility with the data collecting system. Amplifying, filtering, linearizing, calibrating, and isolating signals are some of the signal conditioning techniques.

\vspace{0.5cm}\par
\textbf{Analog-to-Digital Conversion:}
The incessant analog signals from the equipment's sensors are converted into digital data by analog-to-digital converters (ADCs) so that a computer or microcontroller may process it. The analog to digital converters (ADCs) create a digital representation of the signal amplitude by sampling the analog signals at predetermined rates.

\vspace{0.5cm}\par
\textbf{Data Acquisition Hardware:}
The hardware tools used to capture and process the digital data are included in this component. Usually, it contains cards or modules for data acquisition that are connected to the computer system and offer the analog input channels, digital input/output lines, and other characteristics needed for data collection.

\vspace{0.5cm}\par
\textbf{Data Acquisition Software: }
To manage the data acquisition process and operate the hardware for data collection, software programs or libraries are utilized. They offer interfaces for data viewing, analysis, and export and enable setup of sample rates, triggering mechanisms, and data storage choices.

\vspace{0.5cm}\par
\textbf{Data Storage and Analysis: }
The gathered data is kept for further analysis in computer systems, databases, or special storage devices. In order to get actionable insights from the collected data and make well-informed decisions, data analysis may use statistical techniques, signal processing algorithms, visualization tools, and other approaches.


\section{Aims and Objectives}
A multinational corporation, Schneider Electric focuses on providing automation and energy management solutions. Data acquisition systems (DAS) are essential in industrial settings for monitoring and managing operations, improving performance, maintaining quality, and assisting in decision-making. The aim of our project is to develop a ready to run acquisition system by using a multiple channel ADC which will be able for monitoring electrical parameters like voltage, current, temperature, and noise. This system will have high accuracy on battery and low power and will be connected to the cloud through wireless cellular communication technology.


\section{Outline}
Total nine chapters makes up the entire study of this project. Whereas the Corporate Social Responsibility has been discussed in chapter \ref{Corporate}. Some related works and state of arts has been narrated in chapter \ref{Literature}.Fundamental of ADC has been described in chapter \ref{Fundamental of ADC}. Numerous types of wireless IoT protocols and their performance map has been depicted in chapter \ref{Wireless IoT Protocols}. All the continuous development process of Hardware and Software implementation has been demonstrated in chapter \ref{Hardware} and chapter \ref{Software} respectively. The outcome of this project has been summarized in chapter \ref{Result}, and chapter \ref{Conclusion} represents the epilogue of this data acquisition system.




\nomenclature{$DAQ$}{Data acquisition}
\nomenclature{$DAS$}{Data acquisition system}
\nomenclature{$ADC$}{Analog-to-digital converter}

